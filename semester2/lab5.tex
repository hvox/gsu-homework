\definecolor[headingcolor][r=1, b=0.4]
\definecolor[solutionheadercolor][r=0.9, b=0.5]
\setuphead[subject][color=headingcolor, indentnext=yes]
\setuphead[subsubject][color=solutionheadercolor, indentnext=yes]
\setuphead[subject][style={\ss\bfa}, before={\bigskip\bigskip\bigskip\bigskip}, after={}]
\setuphead[subsubject][style={\ss\bfa}, before={}, after={}]

\setuppapersize[A4]
\setuplayout[backspace=2cm, topspace=1cm, header=1cm, bottomspace=1cm,
    footer=1cm, width=middle, height=middle]
\setupbodyfont[libertinus, 14pt]
\mainlanguage[ru]
\setupwhitespace[medium]
\setupindenting[medium, yes]

\definemathmatrix[lmatrix][simplecommand=lmat, left={\left\{\,}, right={\,\right. }]
\definemathmatrix[umatrix][simplecommand=umat, left={\left[\,}, right={\,\right. }]
\definemathmatrix[pmatrix][simplecommand=pmat, left={\left(\,}, right={\,\right) }]
\definemathmatrix[dmatrix][simplecommand=dmat, left={\left|\,}, right={\,\right| }]
\definemathmatrix[gmatrix][simplecommand=grid, left={\left.\,}, right={\,\right. }]
\definemathcommand[arctg][nolop]{\mfunction{arctg}}

\setupindenting[medium,yes]
%\setupformulas[align=right]
\starttext
\setuppagenumbering[state=stop]
\centerline { Гомельский государственный университет имени Франциска Скорины }
\centerline { факультет математики и технологий программирования }
\vfill \vfill
\centerline { Лабораторная работа №5 }
\centerline { «Строение линейного оператора» }
\vfill \vfill
{\leftskip 0.55\hsize \noindent
  Выполнил:\\Хамков Владислав (ПИ-11)\\
  Проверил:\\Васильев Александр Федорович}
\vfill
\centerline { Май 2022 }
\page

\setuppagenumbering[state=start]
\setuppagenumber[number=1]
\setuppagenumbering[location={footer,center}]
\setupfootertexts[\qquad \date \hfill Страница \pagenumber\ / \lastpagenumber \qquad]

\subject {Задание №1}
Найдите собственные значения и собственные векторы линейных операторов,
заданных в некотором базисе линейного пространства над полем ${\Bbb R}$
и над полем ${\Bbb C}$ следующими матрицами.
\startformula
A = \pmat{3, -2; 4, -1} \qquad\qquad B = \pmat{1, -2, -1; -1, 1, 1; 1, 0, -1}
\stopformula
\subsubject {Решение}
TODO


\subject {Задание №2}
Пусть $f$ --- линейный оператор пространства $V$ над полем $P$. Докажите следующее утверждение.
Линейная оболочка любой системы собственных векторов оператора $f$ инвариантна относительно $f$.
\subsubject {Решение}
TODO


\subject {Задание №3}
Какие из матриц линейных операторов в пространстве $V$ над ${\Bbb R}$ можно
привести к диагональному виду путем перехода к новому базису?
Найдите этот базис и соответствующую диагональную матрицу.
\startformula
A = \pmat{1, -1; -4, 4} \qquad
B = \pmat{5, -1, -4; -12, 5, 12; 10, -3, -9} \qquad
C = \pmat{-1, 3, -1; -3, 5, -1; -3, 3, 1}
\stopformula
\subsubject {Решение}
TODO


\subject {Задание №4}
Найдите жорданову нормальную форму матриц.
\startformula
A = \pmat{1, 0; 1, 0} \qquad
B = \pmat{1, -3, 4; 4, -7, 8; 6, -7, 7} \qquad
C = \pmat{4, 1, 1, 1; -1, 2, -1, -1; 6, 1, -1, 1; -6, -1, 4, 2}
\stopformula
\subsubject {Решение}
TODO


\stoptext

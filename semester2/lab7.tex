\definecolor[headingcolor][g=0.6, b=0.4]
\definecolor[solutionheadercolor][g=0.5, b=0.5]
\setuphead[subject][color=headingcolor, indentnext=yes]
\setuphead[subsubject][color=solutionheadercolor, indentnext=yes]
\setuphead[subject][style={\ss\bfa}, before={\bigskip\bigskip\bigskip\bigskip}, after={}]
\setuphead[subsubject][style={\ss\bfa}, before={}, after={}]

\setuppapersize[A4]
\setuplayout[backspace=2cm, topspace=1cm, header=1cm, bottomspace=1cm,
    footer=1cm, width=middle, height=middle]
\setupbodyfont[libertinus, 14pt]
\mainlanguage[ru]
\setupwhitespace[medium]
\setupindenting[medium, yes]

\definemathmatrix[lmatrix][simplecommand=lmat, left={\left\{\,}, right={\,\right. }]
\definemathmatrix[umatrix][simplecommand=umat, left={\left[\,}, right={\,\right. }]
\definemathmatrix[pmatrix][simplecommand=pmat, left={\left(\,}, right={\,\right) }]
\definemathmatrix[dmatrix][simplecommand=dmat, left={\left|\,}, right={\,\right| }]
\definemathmatrix[gmatrix][simplecommand=grid, left={\left.\,}, right={\,\right. }]
\definemathcommand[arctg][nolop]{\mfunction{arctg}}

\setupindenting[medium,yes]
%\setupformulas[align=right]
\starttext
\setuppagenumbering[state=stop]
\centerline { Гомельский государственный университет имени Франциска Скорины }
\centerline { факультет математики и технологий программирования }
\vfill \vfill
\centerline { Лабораторная работа №7 }
\centerline { «Линейные операторы в евклидовом пространстве» }
\vfill \vfill
{\leftskip 0.55\hsize \noindent
  Выполнил:\\Хамков Владислав (ПИ-11)\\
  Проверил:\\Васильев Александр Федорович}
\vfill
\centerline { Июнь 2022 }
\page
\setuppagenumber[number=1]
\setupfootertexts[\qquad \date \hfill Страница \pagenumber\ / \lastpagenumber \qquad]

\subject {Задание №1}
Докажите следующее свойство сопряженных линейных операторов евклидова пространства.
\startformula (f + g)^∗ = f^∗ + g^∗ \stopformula
\subsubject {Доказательство}
Известно, что матрица оператора $f^∗$ в ортонормированном базисе будет транспонированной
по отношению к матрице оператора $f$ в этом базисе. Поэтому уравнение $(f + g)^∗ = f^∗ + g^∗$
можно записать в матричном виде как $(M_f + M_g)^T = M_f^T + M_g^T$, где
$M_f$ --- матрица линейного оператора $f$ в некотором ортонормированном базисе,
а $M_g$ --- матрица линейного оператора $g$ в некотором ортонормированном базисе.
\par
Свойством операции транспонирования является то, что $(A + B)^T = A^T + B^T$.
В частности, верно равенство $(M_f + M_g)^T = M_f^T + M_g^T$.
Следовательно, $(f + g)^∗ = f^∗ + g^∗$, что и требовалось доказать.
	

\subject {Задание №2}
В пространстве ${\Bbb R}^2[x]$ задано скалярное произведение:
$(f, g) = a_0 b_0 + a_1 b_1 + a_2 b_2$, где $f (x) = a_0 + a_1 x + a_2 x^2$,
$g(x) = b_0 + b_1 x + b_2 x^2$. Пусть $F$ — такой линейный оператор
пространства ${\Bbb R}^2[x]$, что $F (f (x))$ = $f (x − a) − f (x − b)$.
Найдите матрицу сопряженного оператора $F^∗$ в базисе $1$, $x$, $x_2$.
\startformula a = 2 \qquad\qquad b = 1 \stopformula
\subsubject {Решение}
Найдём матрицу $M_F$ линейного оператора $F$.
\startformula F(e_1) = F(1) = 1 - 1 = 0 \stopformula
\startformula F(e_2) = F(x) = (x - 2) - (x - 1) = -1 \stopformula
\startformula F(e_3) = F(x^2) = (x - 2)^2 - (x - 1)^2 = 3-2x \stopformula
\startformula M_F = \pmat{
0, -1,  3;
0,  0, -2;
0,  0,  0
} \stopformula
Нетрудно заметить, что базис $1$, $x$, $x_2$ является ортонормированным.
Следовательно матрица сопряженного оператора $F^*$ является транспонированной матрицей
оператора $F$ в этом базисе:
\startformula M_{F^*} = M_F^T = \pmat{0, 0, 0; -1, 0, 0; 3, -2, 0} \stopformula


\subject {Задание №3}
Пусть $e_1$, $e_2$ --- ортонормированный базис евклидова пространства $V$,
$A$ — матрица линейного оператора $f$ в базисе $e′_1 = e_1$, $e′_2 = e_1 + e_2$.
Найдите матрицу оператора $f^∗$ в базисе $e′_1$, $e′_2$.
\startformula A = \pmat{1,-1; 2,2} \stopformula
\subsubject {Решение}
Запишем матрицу перехода от базиса $e_1$, $e_2$ к базису $e`_1$, $e`_2$
\startformula T = \pmat{1,0;1,1} \stopformula
Тогда матрицу $B$ линейного оператора $f$ в базисе $e_1$, $e_2$ можно найти по формуле
$B = T^{-1}AT$. Известно, что матрица $C$ оператора $f^∗$ в ортонормированном базисе будет транспонированной
по отношению к матрице $B$ оператора $f$ в этом базисе. Т.е. $C = B^T$.
\startformula \startalign
\NC C \NC = B^T = \left( T^{-1} A T \right)^T = \NR
\NC\NC = \left(\pmat{1,0;1,1}^{-1} \pmat{1,-1;2,2} \pmat{1,0;1,1} \right)^T = \NR
\NC\NC = \left(\pmat{1,0;-1,1} \pmat{1,-1;2,2} \pmat{1,0;1,1} \right)^T = \NR
\NC\NC = \pmat{0, -1; 4, 3}^T \NR
\NC\NC = \pmat{0, 4; -1, 3} \NR
\stopalign \stopformula
Найдём матрицу $D$ оператора $f^*$ в базисе $e`_1$, $e`_2$
зная матрицу $C$ оператора $f^*$ в базисе $e_1$, $e_2$.
\startformula
D = TCT^{-1} = \pmat{1,0;1,1}\pmat{0,4;-1,3}\pmat{1,0;-1,1} = \pmat{-4,4;-8,7}
\stopformula


\subject {Задание №4}
Найдите матрицу линейного оператора $f$ в ортонормированном базисе
$e_1$, $e_2$, $e_3$, если $f$ переводит векторы $a_1$, $a_2$, $a_3$ в векторы $b_1$, $b_2$, $b_3$ соответственно.
Координаты всех векторов заданы в базисе $e_1$, $e_2$, $e_3$.
\startformula a_1 = (2,3,5) \qquad a_2 = (0,1,2) \qquad a_3 = (1,0,0) \stopformula
\startformula b_1 = (1,1,1) \qquad b_2 = (1,1,-1) \qquad b_3 = (2,1,2) \stopformula
\subsubject {Решение}
\startformula f(a_1) = b_1 \qquad f(a_2) = b_2 \qquad f(a_3) = b_3 \stopformula
\startformula M a_1 = b_1 \qquad M a_2 = b_2 \qquad M a_3 = b_3 \stopformula
\startformula M A = B \qquad \Rightarrow \qquad M = B A^{-1} \stopformula
\startformula \startalign
\NC M_f \NC = \pmat{1,1,2;1,1,1;1,-1,2}\pmat{2,0,1;3,1,0;5,2,0}^{-1} \NR
\NC\NC = \pmat{1,1,2;1,1,1;1,-1,2}\pmat{0,2,-1;0,-5,3;1,-4,2} \NR
\NC\NC = \pmat{2,-11,6;1,-7,4;2,-1,0} \NR
\stopalign \stopformula


\subject {Задание №5}
Определите, является ли ортогональным линейный оператор $φ$ евклидова пространства ${\Bbb R}^n$,
действующий на векторы ортонормированного базиса $e_1$, $e_2$, $e_3$, ... $e_n$ по следующим формулам.
\startformula n = 2 \qquad φ(e_1) = e_1 + e_2 \qquad φ(e_2) = e_1 - e_2 \stopformula
\subsubject {Решение}
Данный оператор не является ортгональным, например для $x=e_1$, $y=e_1$
скалярное произведение не сохраняется.
\startformula
(x, y) = (e_1, e_1) = 1\cdot1 \ne (φ(e_1), φ(e_1)) = (e_1 + e_2, e_1 + e_2) = 1\cdot1 + 1\cdot1 = 2
\stopformula


\subject {Задание №6}
В ортонормированном базисе евклидова пространства ${\Bbb R}^3$ линейный оператор задан матрицей $B$.
Будет ли этот оператор ортогональным?
\startformula B = \pmat{
	\frac12, \frac13, -\frac{\sqrt{2}}{2};
	\frac12, \frac12, \frac{\sqrt{2}}{2};
	\frac{\sqrt{2}}{2}, -\frac{\sqrt{2}}{2}, 0
} \stopformula
\subsubject {Решение}
Данный оператор не является ортгональным, например для $x=(0,1,0)$, $y=(0,1,0)$
скалярное произведение не сохраняется.
\startformula (x, y) = \left((0,1,0), (0,1,0)\right) = 0\cdot0 + 1\cdot1 + 0\cdot0 = 1 \stopformula
\startformula \startalign
\NC (φ(x), φ(y)) \NC = (φ((0, 1, 0)), φ((0, 1, 0))) = \left(B\cdot\pmat{0;1;0}, B\cdot\pmat{0;1;0}\right) \NR
\NC\NC = \left( \frac13, \frac12, - \frac{\sqrt{2}}{2}\right)
	\cdot \left( \frac13, \frac12, - \frac{\sqrt{2}}{2}\right) \NR
\NC\NC =\frac19 + \frac14 + \frac24 = \frac{31}{36} \ne 1 = (x, y)
\stopalign \stopformula



\subject {Задание №7}
Докажите следующее свойство самосопряженных линейных операторов евклидова пространства:
Для любого линейного оператора $f$ евклидова пространства операторы $ff^∗$ и $f^∗f$
являются самосопряженными.
\subsubject {Доказательство}
Докажем, что оператор $ff^*$ являются самосопряженным, т.е.
\startformula \left(ff^*\right)^* = ff^* \stopformula
Запишем это уравнение в матричном виде полагая, что $M_f$ --- матрица линейного оператора $f$
в некотором ортонормированном базисе.
\startformula \left( M_f \cdot M^T_f \thinspace \right)^T = M_f \cdot M^T_f \stopformula
Докажем вышеуказанное уравнение используя свойство транспонирования матриц $(AB)^T = B^T A^T$
\startformula
\left( M_f \cdot M^T_f \thinspace \right)^T = \left(M^T_f\thinspace \right)^T \cdot M^T_f = M_f \cdot M^T_f
\stopformula
Таким образом, $ \left( M_f  \cdot M^T_f \thinspace \right)^T = M_f \thinspace\cdot M^T_f $\thinspace, т.е.
$ \left(ff^*\right)^* = ff^* $, что и требовалось доказать.
\par Аналогично докажем, что $f^∗f$ является самосопряженным оператором, т.е.
$ \left(f^*f\right)^* = f^* f $
\startformula
\left( M^T_f \cdot M_f \thinspace \right)^T = M^T_f \cdot \left(M^T_f\thinspace \right)^T = M^T_f \cdot M_f
\ \Rightarrow\ \left( M^T_f \cdot M_f \thinspace \right)^T = M^T_f \cdot M_f
\ \Rightarrow\ \left(f^*f\right)^* = f^* f
\stopformula



\subject {Задание №8}
Найдите базис ортогонального дополнения линейной оболочки системы векторов $M$ из ${\Bbb R}^4$.
\startformula M = \left{(1, 1, 1, 1),\ (1, 2, 2, −1),\ (1, 0, 0, 3)\right}. \stopformula
\subsubject {Решение}
Вектора ортогонального дополнения линейной оболочки векторов ортогональны векторам данной линейной оболочки. 
Eсли $x \in W^{⊥}$, то $(1,1,1,1) \cdot x = 0$, $(1,2,2,-1) \cdot x = 0$,\ \ $(1,0,0,3) \cdot x = 0 $
\startformula
\pmat{1,1,1,1; 1,2,2,-1; 1,0,0,3}\pmat{x_1;x_2;x_3;x_4} = \pmat{0;0;0}
\quad\Rightarrow\quad
\lmat{x_1 + x_2 + x_3 + x_4 = 0; x_1 + 2x_2 + 2x_3 - x_4 = 0; x_1 + 3x_4 = 0}
\quad\Rightarrow\quad \stopformula \startformula \quad\Rightarrow\quad
\lmat{x_1 = -3a; x_2 = 2a - b; x_3 = b; x_4 = a}
\quad\Rightarrow\quad
\startgmatrix 
\NC e_1 = (-3, 2, 0, 1) \NR
\NC e_2 = (0, -1, 1, 0) \NR
\stopgmatrix
\stopformula
Таким оболочки вектора $(-3, 2, 0, 1)$ и $(0, -1, 1, 0)$ являются базисом
ортогонального дополнения линейной оболочки системы векторов $M$ из ${\Bbb R}^4$.


\subject {Задание №9}
Для данной матрицы $C$ оператора $f$ найдите такую ортогональную матрицу $T$,
что $TCT^{−1}$ --- диагональная матрица. Сделайте проверку.
\startformula C = \pmat{17, -8, 4; -8, 17, -4; 4, -4, 11} \stopformula
\subsubject {Решение}
Данная матрица обладает собственными значениями $27$, $9$, $9$ и собственными векторами
$e'_1 = \left(\frac23, -\frac23, \frac13\right)$,
$e'_2 = \left(\frac{\sqrt5}{5}, 0, -\frac{2\sqrt5}{5} \right)$,
$e'_3 = \left(\frac{4\sqrt5}{15}, \frac{\sqrt5}{3}, \frac{2\sqrt5}{15} \right)$, которые образуют
ортонормированный базис, в котором матрица оператора $f$ является диагональной
матрицей состоящей из собственных значений матрицы $C$.
\par Пусть $T$ --- матрица перехода к $e'_1$, $e'_2$, $e'_3$,
тогда $TCT^{-1}$ будет диагональной матрицей состоящей из собственных значений матрицы $C$.
Матрица $T$ является ортогональной, т.к. она является матрицей перехода от ортонормированного базиса $e_1$, $e_2$, $e_3$
к ортонормированному базису $e'_1$, $e'_2$, $e'_3$.
\startformula
T = \pmat{  \frac23, -\frac23, \frac13;
	\frac{\sqrt5}{5}, 0, -\frac{2\sqrt5}{5};
	\frac{4\sqrt5}{15}, \frac{\sqrt5}{3}, \frac{2\sqrt5}{15} }
\stopformula\startformula\startalign
\NC TCT^{-1} \NC = 
\pmat{\frac23,-\frac23,\frac13;\frac{\sqrt5}{5},0,-\frac{2\sqrt5}{5};\frac{4\sqrt5}{15},\frac{\sqrt5}{3},\frac{2\sqrt5}{15}}
\cdot \pmat{17, -8, 4; -8, 17, -4; 4, -4, 11} \cdot
\pmat{\frac23,-\frac23,\frac13;\frac{\sqrt5}{5},0,-\frac{2\sqrt5}{5};\frac{4\sqrt5}{15},\frac{\sqrt5}{3},\frac{2\sqrt5}{15}}^{-1}
\NR \NC \NC =
\pmat{\frac23,-\frac23,\frac13;\frac{\sqrt5}{5},0,-\frac{2\sqrt5}{5};\frac{4\sqrt5}{15},\frac{\sqrt5}{3},\frac{2\sqrt5}{15}}
\cdot \pmat{17, -8, 4; -8, 17, -4; 4, -4, 11} \cdot
\pmat{\frac23,-\frac23,\frac13;\frac{\sqrt5}{5},0,-\frac{2\sqrt5}{5};\frac{4\sqrt5}{15},\frac{\sqrt5}{3},\frac{2\sqrt5}{15}}^{T}
\NR \NC \NC =
\pmat{\frac23,-\frac23,\frac13;\frac{\sqrt5}{5},0,-\frac{2\sqrt5}{5};\frac{4\sqrt5}{15},\frac{\sqrt5}{3},\frac{2\sqrt5}{15}}
\cdot \pmat{17, -8, 4; -8, 17, -4; 4, -4, 11} \cdot
\pmat{   \frac23,  \frac{\sqrt5}{5},  \frac{4\sqrt5}{15};    
	-\frac23,   0,                \frac{\sqrt5}{3};      
	 \frac13, -\frac{2\sqrt5}{5}, \frac{2\sqrt5}{15} }
\NR \NC \NC =
\pmat{27,0,0;0,9,0;0,0,9}
\stopalign\stopformula
Сделаем проверку, что $TCT^{-1}$ диагональна, а $T$ ортогональна.
\startformula TCT^{-1} = \pmat{27,0,0;0,9,0;0,0,9} \text{ --- очевидно диагональна матрица} \stopformula
Проверим, что $T$ является ортогональной матрицей:
\startformula T\cdot T^{T} =
\pmat{\frac23,-\frac23,\frac13;\frac{\sqrt5}{5},0,-\frac{2\sqrt5}{5};\frac{4\sqrt5}{15},\frac{\sqrt5}{3},\frac{2\sqrt5}{15}} \cdot
\pmat{\frac23,\frac{\sqrt5}{5},\frac{4\sqrt5}{15};-\frac23,0,\frac{\sqrt5}{3};\frac13,-\frac{2\sqrt5}{5},\frac{2\sqrt5}{15}}
= \pmat{1,0,0;0,1,0;0,0,1} = E
\stopformula


\subject {Задание №10}
Найдите собственные значения и ортонормированный базис $e′_1$, $e′_2$
из собственных векторов самосопряженного линейного оператора $φ$,
заданного в некотором ортонормированном базисе $e_1$, $e_2$ матрицей $D$.
Найдите матрицу оператора $φ$ в базисе $e′_1$, $e′_2$.
\startformula D = \pmat{4, -2; -2, 1} \stopformula
\subsubject {Решение}
Данная матрица обладает собственными значениями $5$, $0$ и собственными векторами
$e'_1 = \left( \frac{2\sqrt{5}}{5}, -\frac{\sqrt{5}}{5} \right)$,
$e'_2 = \left( \frac{\sqrt{5}}{5},  \frac{2\sqrt{5}}{5} \right)$, которые образуют
ортонормированный базис, в котором матрица данного оператора является диагональной
матрицей состоящей из собственных значений матрицы $D$.
Т.е. матрицей оператора $φ$ в базисе $e'_1, e'_2$ является матрица
\startformula \pmat{5,0;0,0} \stopformula


\stoptext

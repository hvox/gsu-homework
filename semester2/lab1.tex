\setuppapersize[A4]
\setuplayout[backspace=2cm, topspace=1cm, header=1cm, bottomspace=1cm,
    footer=1cm, width=middle, height=middle]
\setupbodyfont[libertinus, 14pt]
\mainlanguage[ru]
\setuppagenumbering[location={footer,center}]
\definemathmatrix[lmatrix][simplecommand=lmat, left={\left\{\,}, right={\,\right. }]
\definemathmatrix[umatrix][simplecommand=umat, left={\left[\,}, right={\,\right. }]
\definemathmatrix[pmatrix][simplecommand=pmat, left={\left(\,}, right={\,\right) }]
\definemathmatrix[dmatrix][simplecommand=dmat, left={\left|\,}, right={\,\right| }]
\definemathcommand[arctg][nolop]{\mfunction{arctg}}

\setupindenting[medium,yes]
%\setupformulas[align=right]
\starttext
\setuppagenumbering[state=stop]
\centerline { Гомельский государственный университет имени Франциска Скорины }
\centerline { факультет математики и технологий программирования }
\vfill \vfill
\centerline { Лабораторная работа №1 }
\centerline { «Линейные пространства и их начальные свойства» }
\vfill \vfill
{\leftskip 0.55\hsize \noindent
  Выполнил:\\Хамков Владислав (ПИ-11)\\
  Проверил:\\Васильев Александр Федорович}
\vfill
\centerline { Март 2022 }
\page

\setuppagenumbering[state=start]
\setuppagenumber[number=1]
\centerline {Задание №1}
Является ли действительным линейным пространством с операциями, определенными в $\bf{R}^n$,
множество всех тех строк $(α_1, α_2, ... , α_n)$, которые удовлетворяют следующему условию:
$α_1 = 0$? \\


Проверим является ли операции $+$ и $\cdot$ замкнутыми на вышеуказанном
множестве:
\startformula
\left(0, α_2, α_3, ... , α_n\right) +
\left(0, β_2, β_3, ... , β_n\right) =
\left(0, α_2 + β_2, α_3 + β_3, ... , α_n + β_n\right) =
\left(0, γ_2, γ_3, ... , γ_n\right)
\stopformula
\startformula
\left(0, α_2, α_3, ... , α_n\right) \cdot β =
\left(0 \cdot β, α_2 \cdot β, α_3 \cdot β, ... , α_n \cdot β\right) =
\left(0, γ_2, γ_3, ... , γ_n\right)
\stopformula
Т.е. операции $+$ и $\cdot$ являются замкнутыми на множестве все тех 
строк $(α_1, α_2, ... , α_n)$, которые удовлетворяют следующему условию:
$α_1 = 0$. Таким образом, остальные требуемые свойства для линейного пространства
выполняются по наследству от $\bf{R}^n$.
\vfil



\centerline {Задание №2}
Является ли комплексным линейным пространством с операциями сложения
многочленов и умножения многочленов на комплексное число множество
всех тех многочленов $f(x) ∈ C_n[x]$, которые удовлетворяют следующему
условию: $f(0) = 0$?


Проверим является ли операции $+$ и $\cdot$ замкнутыми на вышеуказанном
множестве. Пусть $h(x) = f(x) + g(x)$, тогда верно следующее утверждение:
\startformula
h(0) = f(0) + g(0) = 0 + 0 = 0
\stopformula
Таким образом, операция $+$ замкнута на вышеуказанном множестве.
Пусть $h(x) = f(x) \cdot β$, тогда верно следующее утверждение:
\startformula
h(0) = f(0) \cdot β = 0 \cdot β = 0
\stopformula
Таким образом, операция $\cdot$ замкнута на вышеуказанном множестве.
Поэтому вышеуказанное множество является линейным пространством по
наследству от $C_n[x]$.
\page
\centerline {Задание №3}
Докажите, что множество решений однородной системы линейных уравнений
образует линейное пространство!
\startformula
\lmat{
	x_1 + 4x_2 + x_3 + 2x_4 = 0;
	2x_1 + 7x_2 - 2x_3 + 0 = 0;
	-x_1 + 3x_2 + 0 - x_4 = 0
}
\stopformula
Решим эту систему уравнений методом Гауса:
\startformula
\pmat{1,4,1,2;2,7,-2,0;-1,3,0,-1} \rightarrow
\pmat{1,4,1,2;0,-1,-4,-4;0,0,-27,-27} \rightarrow
\lmat{ x_1 = 29 x_4; x_2 = -8 x_4; x_3 = x_4 }
\stopformula
Таким образом мы вяснили, что множеством решений является любая строка
вида $(29α, -8α, α, α)$, где $α \in \bf{R}$.
Заметим, что это множество является подмножетсвом множества $\bf{R}^4$.
Теперь покажем, что операции $+$ и $\cdot$ замкнуты на вышеуказанном
множестве.
\startformula
\left(29α, -8α, α, α\right) + \left(29β, -8β, β, β\right) =
\left(29α + 29β, -8α -8β, α + β, α + β\right) =
\stopformula
\startformula
= \left(29(α + β), -8(α + β), α + β, α + β\right) =
\left(29γ, -8γ, γ, γ\right)
\stopformula
Аналогично покажем замкнутость операции умножения на элемент поля $\bf{R}$.
\startformula
\left(29α, -8α, α, α\right) \cdot β =
\left(29αβ, -8αβ, αβ, αβ\right) =
\left(29γ, -8γ, γ, γ\right)
\stopformula
Таким образом, вышеуказанное множество является линейным пространством по
наследству от $\bf{R}^4$.
\vfil




\centerline {Задание №4}
Являются ли векторы $a_1$ , $a_2$ , $a_3$ , $a_4$ пространства $\bf{R}^4$
линейно зависимыми? В случае утвердительного ответа
найдите нетривиальную линейную комбинацию, равную
нулю.
\startformula a_1 = \left(1, 1, -1, 0\right) \quad\quad\quad
a_2 = \left(-1, 2, 2, -3\right) \stopformula
\startformula a_3 = \left(-2, 1, -1, 4\right) \quad\quad\quad
a_4 = \left(-1, 1, 2, -2\right) \stopformula
\page
\par
Для определения являются ли вышеуказанные векторы линейно независимыми
нам следует найти определитель матрицы составленной из их координат, и
сравнить его с нулём.
\startformula 
\dmat{
	1, 1, -1, 0;
	-1, 2, 2, -3;
	-2, 1, -1, 4;
	-1, 1, 2, -2
} = -7 < 0
\stopformula
Определитель матрицы оказался ненулевым, поэтому векторы --- линейно независимые.
\vfil




\centerline {Задание №5}
Являются ли линейно независимыми векторы $z_1$ и $z_2$ из пространства $\bf{C}$.
\startformula
z_1 = -3 + i
\quad \quad \quad
z_2 = 2 + 3i
\stopformula
Для определения являются ли вышеуказанные векторы линейно независимыми
нам следует найти определитель матрицы составленной из их координат, и
сравнить его с нулём.
\startformula 
\dmat{
	-3, 1;
	2, 3
} = -9 - 2 = -11 < 0
\stopformula
Определитель матрицы оказался ненулевым, поэтому векторы --- линейно зависимые.
\vfil




\centerline {Задание №6}
Является ли линейно зависимой система векторов $\sin{x}, \cos{x}$ пространства $C_{[a, b]}$?
\par
Для решение этой задачи предположим, что вышеуказанные вектора являются линейно зависимыми
и попробуем найти такие ненулевые коэффициенты $α_1$ и $α_2$, чтобы $α_1\sin{x} + α_2\cos{x} = 0$.
\\
При $x=0$, мы получаем $α_1\sin{x} + α_2\cos{x} = α_2 = 0$ \\
При $x=\frac{π}{2}$, мы получаем $α_1\sin{x} + α_2\cos{x} = α_1 = 0$ \\
Таким образом, мы выяснили, что только нулевые $α_1$ и $α_2$ являются решением уравнения
$α_1\sin{x} + α_2\cos{x} = 0$. Т.е. вышеуказанные вектора являются линейно независимыми.
\page



\centerline {Задание №7}
Являются ли векторы $f_1(x)$, $f_2(x)$, $f_3(x)$ линейного
пространства $\bf{R}_2[x]$ линейно зависимыми?
В случае утвердительного ответа нужно найти нетривиальную линейную
комбинацию, равную нулю.
\startformula
f_1(x) = 6x + 9
\quad \quad \quad
f_2(x) = x^2 - 8x +12
\quad \quad \quad
f_3(x) = 2x^2 + 1
\stopformula
\par
Для решение этой задачи предположим, что вышеуказанные вектора являются линейно зависимыми
и попробуем найти их нетривиальную линейную комбинацию, равную нулю.
Для этого мы подставим вместо $x$ какие-нибудь конкретные значения.
\startformula
f_1(0) = 9
\quad \quad \quad
f_2(0) = 12
\quad \quad \quad
f_3(0) = 1
\stopformula
\startformula
f_1(1) = 15
\quad \quad \quad
f_2(1) = 5
\quad \quad \quad
f_3(1) = 3
\stopformula
\startformula
f_1(-1) = 3
\quad \quad \quad
f_2(-1) = 21
\quad \quad \quad
f_3(-1) = 3
\stopformula
Используя вышеуказанные значения функций можно составить систему:
\startformula
\lmat{
	9α_1 + 12α_2 + α_3 = 0;
	15α_1 + 5α_2 + 3α_3 = 0;
	3α_1 + 21α_2 + 3α_3 = 0
}
\stopformula
Это система является однородной. Поэтому, перед её решением следует найти
определитель её матрицы для того, чтобы выяснить существуеют ли у неё
нетривиальные решения.
\startformula
\pmat{
	9, 12, 1;
	15, 5, 3;
	3, 21, 3
} = -564 < 0
\stopformula
Таким образом мы выяснили, что у вышеуказанных векторов не существует
нетривиальной линейной комбинации, равной нулю.
Т.е. они являются линейно независимыми.
\page




\centerline {Задание №8}
Найдите все значения $λ$, при которых в пространстве
$\bf{R}^4$ вектор $c$ линейно выражается через векторы $a_1$, $a_2$, $a_3$.
\startformula
a_1 = (-1,0,2,5)
\quad\quad\quad
a_2 = (1,-1,2,0)
\quad\quad\quad
a_3 = (2,0,1,5)
\quad\quad\quad
c = (-1,-2,λ,3)
\stopformula
\par
Для выполнения этого задания решим следующее уравнение:
\startformula
β_1 a_1 + β_2 a_2 + β_3 a_3 = c
\stopformula
В матричном виде оно будет выглядеть вот так:
\startformula
\lmat{
	-β_1 + β_2 + 2β_3 = -1;
	0 - β_2 + 0 = -2;
	2β_1 + 2β_2 + β_3 = λ; 
	5β_1 + 0 + 5β_3 = 3
}
\rightarrow
\lmat{
	β_1 = \frac75;
	β_2 = 2;
	β_3 = -\frac45;
	γ = 6
}
\stopformula
Таким образом, ответ: $γ = 6$.
\vfil




\centerline {Задание №9}
Докажите, что линейные пространства $V$ и $W$ изоморфны.
\startformula
{V = \bf{C}\ \text{над}\ \bf{R} }
\quad\quad\quad
W = \bf{R} ^ 2
\stopformula
\par
Установим соответствие между векторами данных пространств
следующим образом:
\startformula
f: a + bi \leftrightarrow (a, b)
\stopformula
Тогда верны следующие равенства:
\startformula
f((a_1 + b_1i) + (a_2 + b_2i)) = f((a_1 + a_2) + (b_1 + b_2)i) = (a_1 + a_2, b_1 + b_2) =
\stopformula \startformula
= (a_1, b_1) + (a_2, b_2) = f(a_1 + b_1i) + f(a_2 + b_2i)
\stopformula
\startformula
f(λ \cdot (a + bi)) = f(λa + λbi) = (λa, λb) = λ(a, b) = λ f(a + bi)
\stopformula
Из вышеуказанного следует, что $f$ является изоморфизмом линейных пространств $V$ и $W$.

\stoptext

\definecolor[headingcolor][r=1, b=0.4]
\definecolor[solutionheadercolor][r=0.9, g=0.5]
\setuphead[subject][color=headingcolor, indentnext=yes]
\setuphead[subsubject][color=solutionheadercolor, indentnext=yes]
\setuphead[subject][style={\ss\bfa}, before={\bigskip\bigskip\bigskip\bigskip}, after={}]
\setuphead[subsubject][style={\ss\bfa}, before={}, after={}]

\setuppapersize[A4]
\setuplayout[backspace=2cm, topspace=1cm, header=1cm, bottomspace=1cm,
    footer=1cm, width=middle, height=middle]
\setupbodyfont[libertinus, 14pt]
\mainlanguage[ru]
\setupwhitespace[medium]
\setupindenting[medium, yes]

\definemathmatrix[lmatrix][simplecommand=lmat, left={\left\{\,}, right={\,\right. }]
\definemathmatrix[umatrix][simplecommand=umat, left={\left[\,}, right={\,\right. }]
\definemathmatrix[pmatrix][simplecommand=pmat, left={\left(\,}, right={\,\right) }]
\definemathmatrix[dmatrix][simplecommand=dmat, left={\left|\,}, right={\,\right| }]
\definemathmatrix[gmatrix][simplecommand=grid, left={\left.\,}, right={\,\right. }]
\definemathcommand[arctg][nolop]{\mfunction{arctg}}

\setupindenting[medium,yes]
%\setupformulas[align=right]
\starttext
\setuppagenumbering[state=stop]
\centerline { Гомельский государственный университет имени Франциска Скорины }
\centerline { факультет математики и технологий программирования }
\vfill \vfill
\centerline { Лабораторная работа №8 }
\centerline { «Квадратичные формы» }
\vfill \vfill
{\leftskip 0.55\hsize \noindent
  Выполнил:\\Хамков Владислав (ПИ-11)\\
  Проверил:\\Васильев Александр Федорович}
\vfill
\centerline { Июнь 2022 }
\page
\setuppagenumber[number=1]
\setupfootertexts[\qquad \date \hfill Страница \pagenumber\ / \lastpagenumber \qquad]

\subject {Задание №1}
Запишите матрицы квадратичных форм $F_1$, $F_2$. Вычислите ранги этих квадратичных форм.
\startformula F_1(x, y, z) = x^2 - 2xy + 2z^2 + 4yz + 5z^2 \stopformula
\startformula F_2(x_1, x_2, x_3, x_4) = x_1 x_2 + x_1 x_3 + x_1 x_4 + x_2 x_3 + x_2 x_4 \stopformula
\subsubject {Решение}
\startformula
M_{F_1} = \pmat{1, -1, 0; -1, 2, 2; 0, 2, 5} \qquad \qquad
M_{F_2} = \pmat{
	0, \frac12, \frac12, \frac12;
	\frac12, 0, \frac12, \frac12;
	\frac12, \frac12, 0, 0;
	\frac12, \frac12, 0, 0 }
\stopformula
\startformula \text{rang}\ F_1 = \text{rang}\ M_{F_1} = \text{rang}\ \pmat{1, -1, 0; -1, 2, 2; 0, 2, 5} = 3 \stopformula
\startformula
\text{rang}\ F_2 = \text{rang}\ M_{F_2} = \text{rang}\
\pmat{ 0, \frac12, \frac12, \frac12; \frac12, 0, \frac12, \frac12; \frac12, \frac12, 0, 0; \frac12, \frac12, 0, 0 } = 3
\stopformula


\subject {Задание №2}
Методом Лагранжа приведите квадратичные формы $G_1$ и $G_2$ к каноническому виду.
Укажите невырожденное линейное преобразование переменных, приводящее к этому виду. Сделайте проверку.
\startformula G_1 = x_1^2 − 2x_2^2 + x_3^2 + 2x_1 x_2 + 4x_1 x_3 + 2x_2 x_3 \stopformula
\startformula G_2 = 2x_1 x_3 − 4x_2 x_3 \stopformula
\subsubject {Решение}
Приведём квадратичную форму $G_1$ к каноническому виду.
\startformula \startalign
\NC G_1 \NC = x_1^2 − 2x_2^2 + x_3^2 + 2x_1 x_2 + 4x_1 x_3 + 2x_2 x_3 \NR
\NC\NC = (x_1 + x_2 + 2x_3)^2 - 3x_2^2 - 2x_2x_3 - 3x_3^2 \NR
\NC\NC = y_1^2 - 3x_2^2 - 2x_2x_3 - 3x_3^2 \NR
\NC\NC = y_1^2 - 3\left(x_2 + \frac13x_3\right)^2 - \frac83x_3^2 \NR
\NC\NC = y_1^2 - 3y_2^2 - \frac83y_3^2 \NR
\stopalign \stopformula
Этим заменам переменных соответствует линейное преобразование:
\startformula \startcases
y_1 = x_1 + x_2 + 2x_3 \NR
y_2 = x_2 + \frac13x_3 \NR
y_3 = x_3 \NR
\stopcases \stopformula
Этому линейному преобразованию соответствую следующая матрица:
\startformula \pmat{1,1,2; 0,1,\frac13; 0,0,3} \stopformula
Приведём квадратичную форму $G_2$ к каноническому виду.
\startformula \startalign
\NC G_2 \NC = 2 x_1 x_3 - 4 x_2 x_3 \NR
\NC\NC =  2 (z_1 - z_2) (z_1 + z_2) - 4 x_2 (z_1 + z_2) \NR
\NC\NC = -4 x_2 z_1 + 2 z_1^2 - 4 x_2 z_2 - 2 z_2^2 \NR
\NC\NC = 2(z_1 - x_2)^2 - 2x_2^2 - 4x_2z_2 - 2z_2^2 \NR
\NC\NC = 2y_1^2 - 2x_2^2 - 4x_2z_2 - 2z_2^2 \NR
\NC\NC = 2y_1^2 - 2(x_2 + z_2)^2 \NR
\NC\NC = 2y_1^2 - 2y_2^2 \NR
\NC\NC = 2y_1^2 - 2y_2^2 + 0y_3^2 \NR
\stopalign \stopformula
Использованным заменам переменных соответствует следующее линейное преобразование:
\startformula \startcases
y_1 = z_1 - x_2 = \frac{x_1}2 - x_2 + \frac{x_3}2 \NR
y_2 = x_2 + z_2 = -\frac{x_1}2 + x_2 + \frac{x_3}2 \NR
y_3 = x_1 \NR
\stopcases \stopformula
Этому линейному преобразованию соответствую следующая матрица:
\startformula \pmat{\frac12,-1,\frac12; -\frac12,1,\frac12; 1,0,0} \stopformula
Проверим, что найденные линейные преобразования являются невырожденными:
\startformula
\dmat{1,1,2; 0,1,\frac13; 0,0,3} = 3
\qquad\qquad
\dmat{\frac12,-1,\frac12; -\frac12,1,\frac12; 1,0,0} = -1
\stopformula

\subject {Задание №3}
Определите индекс инерции квадратичных форм $F_1$, $G_1$, $F_2$, $G_2$.
\startformula F_1(x, y, z) = x^2 - 2xy + 2z^2 + 4yz + 5z^2 \stopformula
\startformula G_1 = x_1^2 − 2x_2^2 + x_3^2 + 2x_1 x_2 + 4x_1 x_3 + 2x_2 x_3 \stopformula
\startformula F_2(x_1, x_2, x_3, x_4) = x_1 x_2 + x_1 x_3 + x_1 x_4 + x_2 x_3 + x_2 x_4 \stopformula
\startformula G_2 = 2x_1 x_3 − 4x_2 x_3 \stopformula
\subsubject {Решение}
Для нахождения индексов инерции запишем матрицы данных квадратичных форм.
\startformula
M_{F_1} = \pmat{1,-1,0;-1,0,2;0,2,5}
\qquad \qquad
M_{G_1} = \pmat{1,1,2;1,-2,1;2,1,1}
\stopformula
\startformula
M_{F_2} = \pmat{0,\frac12,\frac12,\frac12; \frac12,0,\frac12,\frac12; \frac12,\frac12,0,0; \frac12,\frac12,0,0}
\qquad \qquad
M_{G_2} = \pmat{0,0,1; 0,0,-2; 1,-2,0}
\stopformula
Найдём индексы инерции квадратичных форм как ранги полученных матриц.
\startformula
 \text{rang}\ \pmat{1,-1,0;-1,0,2;0,2,5} = 3
\qquad \qquad
 \text{rang}\ \pmat{1,1,2;1,-2,1;2,1,1} = 3
\stopformula
\startformula
 \text{rang}\ \pmat{0,\frac12,\frac12,\frac12; \frac12,0,\frac12,\frac12; \frac12,\frac12,0,0; \frac12,\frac12,0,0} = 3
\qquad \qquad
 \text{rang}\ \pmat{0,0,1; 0,0,-2; 1,-2,0} = 2
\stopformula


\subject {Задание №4}
Найдите ортогональную матрицу $T$ такую, что $T AT^{−1}$ --- диагональная матрица,
где $A$ --- матрица квадратичной формы $F$. Сделайте проверку.
\startformula F = 17x_1^2 + 17x_2^2 + 11x_3^2 − 16x_1 x_2 + 8x_1 x_3 − 8x_2 x_3 \stopformula
\subsubject {Решение}
Запишем матрицу данной квадратичной формы.
\startformula A = \pmat{17, -8, 4; -8, 17, -4; 4, -4, 11} \stopformula
Методом пристального взгляда я выяснил, что я уже выполнял это задание для этой матрицы в прошлой лабораторной работе.
Используя специальную технику CTRL-C / CTRL-V я получаю матрицу $T$:
\startformula
T = \pmat{  \frac23, -\frac23, \frac13;
	\frac{\sqrt5}{5}, 0, -\frac{2\sqrt5}{5};
	\frac{4\sqrt5}{15}, \frac{\sqrt5}{3}, \frac{2\sqrt5}{15} }
\stopformula
Сделаем проверку, что $TAT^{-1}$ диагональна, а $T$ ортогональна.
\startformula\startalign
\NC TAT^{-1} \NC = 
\pmat{\frac23,-\frac23,\frac13;\frac{\sqrt5}{5},0,-\frac{2\sqrt5}{5};\frac{4\sqrt5}{15},\frac{\sqrt5}{3},\frac{2\sqrt5}{15}}
\cdot \pmat{17, -8, 4; -8, 17, -4; 4, -4, 11} \cdot
\pmat{\frac23,-\frac23,\frac13;\frac{\sqrt5}{5},0,-\frac{2\sqrt5}{5};\frac{4\sqrt5}{15},\frac{\sqrt5}{3},\frac{2\sqrt5}{15}}^{-1}
\NR \NC \NC =
\pmat{\frac23,-\frac23,\frac13;\frac{\sqrt5}{5},0,-\frac{2\sqrt5}{5};\frac{4\sqrt5}{15},\frac{\sqrt5}{3},\frac{2\sqrt5}{15}}
\cdot \pmat{17, -8, 4; -8, 17, -4; 4, -4, 11} \cdot
\pmat{\frac23,-\frac23,\frac13;\frac{\sqrt5}{5},0,-\frac{2\sqrt5}{5};\frac{4\sqrt5}{15},\frac{\sqrt5}{3},\frac{2\sqrt5}{15}}^{T}
\NR \NC \NC =
\pmat{\frac23,-\frac23,\frac13;\frac{\sqrt5}{5},0,-\frac{2\sqrt5}{5};\frac{4\sqrt5}{15},\frac{\sqrt5}{3},\frac{2\sqrt5}{15}}
\cdot \pmat{17, -8, 4; -8, 17, -4; 4, -4, 11} \cdot
\pmat{   \frac23,  \frac{\sqrt5}{5},  \frac{4\sqrt5}{15};    
	-\frac23,   0,                \frac{\sqrt5}{3};      
	 \frac13, -\frac{2\sqrt5}{5}, \frac{2\sqrt5}{15} }
\NR \NC \NC =
\pmat{27,0,0;0,9,0;0,0,9}
\stopalign\stopformula
\startformula TCT^{-1} = \pmat{27,0,0;0,9,0;0,0,9} \text{ --- диагональная матрица} \stopformula
Проверим, что $T$ является ортогональной матрицей:
\startformula T\cdot T^{T} =
\pmat{\frac23,-\frac23,\frac13;\frac{\sqrt5}{5},0,-\frac{2\sqrt5}{5};\frac{4\sqrt5}{15},\frac{\sqrt5}{3},\frac{2\sqrt5}{15}} \cdot
\pmat{\frac23,\frac{\sqrt5}{5},\frac{4\sqrt5}{15};-\frac23,0,\frac{\sqrt5}{3};\frac13,-\frac{2\sqrt5}{5},\frac{2\sqrt5}{15}}
= \pmat{1,0,0;0,1,0;0,0,1} = E
\stopformula



\subject {Задание №5}
Найдите ортогональное преобразование переменных,
приводящее квадратичную форму $F(x_1, x_2, x_3)$ к каноническому виду.
\startformula F = 8x_1^2 − 7x_2^2 + 8x_3^2 + 8x_1 x_2 − 2x_1 x_3 + 8x_2 x_3 \stopformula
\subsubject {Решение}
Запишем матрицу данной квадратичной формы.
\startformula A = \pmat{8,  4, -1; 4, -7,  4; -1,  4,  8} \stopformula
Данная матрица обладает собственными значениями $-9$, $9$, $9$ и собственными векторами
$(1, -4, 1)$, $(1, 0, -1)$, $(0, 1, 4)$ соответственно.
После нормализации и ортгонализации этих векторов получаем ортонормированный базис:
\startformula
\left(\frac16 \sqrt2,\ -\frac23\sqrt2,\ \frac16\sqrt2\right)
\qquad
\left(\frac12 \sqrt2,\ 0,\ -\frac12\sqrt2\right)
\qquad
\left(0,\ \frac1{17}\sqrt{17},\ \frac4{17}\sqrt{17}\right)
\stopformula
Теоретически, в этом базисе квадратичная форма $F$ будет иметь канонический вид с собственными значениями в качестве коэфициентов.
Составим матрицу перехода к этому базису.
\startformula
T = \pmat{
	\frac16 \sqrt2,-\frac23\sqrt2,\frac16\sqrt2;
	\frac1{\sqrt{2}},0,-\frac1{\sqrt2};
	0,\frac1{\sqrt{17}},\frac4{\sqrt{17}}}
\stopformula
Сделаем проверку, что $TAT^{-1}$ диагональна, а $T$ ортогональна.
\startformula \startalign
\NC TAT^{-1} \NC =
	\pmat{ \frac16 \sqrt2,-\frac23\sqrt2,\frac16\sqrt2; \frac1{\sqrt{2}},0,-\frac1{\sqrt2}; 0,\frac1{\sqrt{17}},\frac4{\sqrt{17}}}
	\cdot
	\pmat{8,  4, -1; 4, -7,  4; -1,  4,  8}
	\cdot
	\pmat{ \frac16\sqrt2, \frac12\sqrt2, \frac23; -\frac23\sqrt2, 0, \frac13; \frac16\sqrt2, - \frac12\sqrt2, \frac23 }
\NR \NC \NC = \pmat{-9,0,0; 0,9,0; 0,0,9} \NR
\stopalign \stopformula



\subject {Задание №6}
Найдите нормальный вид над полем ${\Bbb R}$ и над полем ${\Bbb C}$,
а также невырожденное линейное преобразование переменных,
приводящее к этому виду, для квадратичных форм $F_1$ и $F_2$.
\startformula F_1 = 17x_1^2 + 17x_2^2 + 11x_3^2 − 16x_1 x_2 + 8x_1 x_3 − 8x_2 x_3 \stopformula
\startformula F_2 = 8x_1^2 − 7x_2^2 + 8x_3^2 + 8x_1 x_2 − 2x_1 x_3 + 8x_2 x_3 \stopformula
\subsubject {Решение}
Найдите нормальный вид $F_1$ над полем ${\Bbb R}$.
Вспомним, что $F_1$ линейным преобразованием $T$ приводится к канонической форме $27y_1 + 9y_2 + 9y_3$.
\startformula
T = \pmat{
	\frac23, -\frac23, \frac13;
	\frac{\sqrt5}{5}, 0, -\frac{2\sqrt5}{5};
	\frac{4\sqrt5}{15}, \frac{\sqrt5}{3}, \frac{2\sqrt5}{15} }
\stopformula
Полсе этого заменой $z_1 = 3\sqrt{3}y_1$, $z_2=3y_2$, $z_3=3y_3$ получаем нормальный вид квадратичной формы $F_1$:
\startformula F_1 = z_1^2 + z_2^2 + z_3^2 \stopformula
Найдём линейное преобразование переменных, соответствующее переходу от переменных $x_1$, $x_2$, $x_3$ к переменным $z_1$, $z_2$, $z_3$:
\startformula \startalign
\NC T' \NC = \pmat{3\sqrt3,0,0; 0,3,0; 0,0,3} \cdot T \NR
\NC \NC =
	\pmat{\frac1{3\sqrt3},0,0; 0,\frac13,0; 0,0,\frac13} \cdot
	\pmat{ \frac23, -\frac23, \frac13; \frac{\sqrt5}{5}, 0, -\frac{2\sqrt5}{5}; \frac{4\sqrt5}{15}, \frac{\sqrt5}{3}, \frac{2\sqrt5}{15} }
\NR\NC\NC = \pmat{
	\frac{2\sqrt3}{27}, -\frac{2\sqrt3}{27}, \frac1{27}\sqrt3;
	\frac{\sqrt5}{15}, 0, -\frac{2\sqrt5}{15};
	\frac{4\sqrt5}{45}, \frac{\sqrt5}9, \frac{2\sqrt5}{45}} \NR
\stopalign \stopformula
Поскольку все коэфициенты в этом нормальном виде равны единицам,
то $F_1$ обладает данной нормальной формой и над полем ${\Bbb R}$ и над ${\Bbb C}$.
\par Найдите нормальный вид $F_2$ над полем ${\Bbb R}$.
Вспомним, что $F_1$ линейным преобразованием $T$ приводится к канонической форме $-9y_1 + 9y_2 + 9y_3$.
\startformula T = \pmat{
	\frac16 \sqrt2,-\frac23\sqrt2,\frac16\sqrt2;
	\frac1{\sqrt{2}},0,-\frac1{\sqrt2};
	0,\frac1{\sqrt{17}},\frac4{\sqrt{17}}}
\stopformula
Полсе этого заменой $z_1 = 3y_1$, $z_2=3y_2$, $z_3=3y_3$ получаем нормальный вид квадратичной формы $F_2$:
\startformula F_2 = -z_1^2 + z_2^2 + z_3^2 \stopformula
Найдём линейное преобразование переменных, соответствующее переходу от переменных $x_1$, $x_2$, $x_3$ к переменным $z_1$, $z_2$, $z_3$:
\startformula \startalign
\NC T' \NC = \pmat{\frac13,0,0; 0,\frac13,0; 0,0,\frac13} \cdot T \NR
\NC \NC =
	\pmat{\frac13,0,0; 0,\frac13,0; 0,0,\frac13} \cdot
	\pmat{ \frac16 \sqrt2,-\frac23\sqrt2,\frac16\sqrt2; \frac1{\sqrt{2}},0,-\frac1{\sqrt2}; 0,\frac1{\sqrt{17}},\frac4{\sqrt{17}}}
\NR\NC\NC = \pmat{
	\frac{\sqrt2}{18}, -\frac{2\sqrt2}9, \frac{\sqrt2}{18};
	\frac{\sqrt2}6, 0, -\frac{\sqrt2}6;
	\frac29, \frac19, \frac29 } \NR
\stopalign \stopformula
Таким образом квадратичная форма $F_2$ обладает нормальным видом $-z_1^2 + z_2^2 + z_3^2$ над полем ${\Bbb R}$.
При помощи замены переменной $w = iz_1$ получаем нормальным вид $w^2 + z_2^2 + z_3^2$ над полем ${\Bbb C}$.
Найдём линейное преобразование переменных, соответствующее переходу от переменных $x_1$, $x_2$, $x_3$ к переменным $w$, $z_2$, $z_3$:
\startformula \startalign
\NC T'' \NC = \pmat{i,0,0; 0,1,0; 0,0,1} \cdot T' \NR
\NC \NC =
	\pmat{i,0,0; 0,1,0; 0,0,1} \cdot
	\pmat{\frac{\sqrt2}{18}, -\frac{2\sqrt2}9, \frac{\sqrt2}{18}; \frac{\sqrt2}6, 0, -\frac{\sqrt2}6; \frac29, \frac19, \frac29 }
\NR\NC\NC = \pmat{
	\frac{\sqrt2}{18}i, -\frac{2\sqrt2}9i, \frac{\sqrt2}{18}i;
	\frac{\sqrt2}6, 0, -\frac{\sqrt2}6;
	\frac29, \frac19, \frac29 } \NR
\stopalign \stopformula



\subject {Задание №7}
Вычислите положительный и отрицательный индексы инерции квадратичных форм $G_1$, $G_2$.
Являются ли эти формы положительно определенными (отрицательно определенными)?
\startformula G_1 = x_1^2 − 2x_2^2 + x_3^2 + 2x_1 x_2 + 4x_1 x_3 + 2x_2 x_3 \stopformula
\startformula G_2 = 2x_1 x_3 − 4x_2 x_3 \stopformula
\subsubject {Решение}
Для нахождения положительных и отрицательных индексов инерции данных квадратичных форм,
найдём собственные значения матриц им соответствующих.
Ведь собственные значения являются коэфициентами в некотором каноническом виде этих квадратичных форм.
\startformula M_{G_1} = \pmat{1,1,2;1,-2,1;2,1,1} \qquad \qquad M_{G_2} = \pmat{0,0,1; 0,0,-2; 1,-2,0} \stopformula
При помощи калькулятора я выяснил, что собственными значениями матрицы $M_{G_1}$ являются $-1$, $-2.3722813...$, $3.37228...$,
т.е. положительный индекс инерции квадратичной формы $G_1$ равен $1$, а отрицательный --- $2$.
Собственными значениями матрицы $M_{G_2}$ являются $0$, $-2.23606...$, $2.23606...$,
т.е. положительный и отрицательный индексы инерции квадратичной формы $G_2$ равены $1$ и $1$.


\subject {Задание №8}
При каких значениях $λ$ квадратичная форма $F$ является положительно (отрицательно) определенной?
\startformula F = 2x_1^2 + x_2^2 + 3x_3^2 + 2λx_1 x_2 + 2x_1 x_3 \stopformula
\subsubject {Решение}
Составим матрицу данной квадратичной формы:
\startformula \pmat{2,λ,1; λ,1,0; 1,0,3} \stopformula
Данная матрица обладает тремя главными минорами:
\startformula 
\Delta_1 = λ
\qquad
\Delta_2 = \dmat{2,λ; λ,1} = 2 - λ^2
\qquad
\Delta_3 = \dmat{2,λ,1; λ,1,0; 1,0,3} = -3λ^2 + 5
\stopformula
Найдём такие значения $λ$, при которых квадратичная форма $F$ является положительно определенной.
\startformula
\Delta_1 = λ > 0 \qquad
\Delta_2 = 2 - λ^2 > 0 \Rightarrow |λ| < \sqrt2 \qquad
\Delta_3 = -3λ^2 + 5 > 0 \Rightarrow |λ| < \sqrt{\frac53}
\stopformula
Таким образом, квадратичная форма $F$ является положительно определенной при $0 < λ < \sqrt{\frac53}$
\par Найдём такие значения $λ$, при которых квадратичная форма $F$ является отрицательно определенной.
\startformula
\Delta_1 = λ < 0 \qquad
\Delta_2 = 2 - λ^2 > 0 \Rightarrow |λ| < \sqrt2 \qquad
\Delta_3 = -3λ^2 + 5 < 0 \Rightarrow |λ| > \sqrt{\frac53}
\stopformula
Таким образом, квадратичная форма $F$ является отрицательно определенной при $\sqrt{\frac53} < λ < \sqrt2$


\stoptext

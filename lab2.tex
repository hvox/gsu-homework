\setuppapersize[A4]
\setuplayout[backspace=2cm, topspace=1cm, header=1cm, bottomspace=1cm, footer=1cm, width=middle, height=middle]
\setupbodyfont[libertinus, 14pt]
\mainlanguage[ru]
\setuppagenumbering[location={footer,center}]

\definemathmatrix[pmatrix][matrix:parentheses][simplecommand=pmatrix]
\setupformulas[align=right]

\starttext
\setuppagenumbering[state=stop]
\centerline { Гомельский государственный университет имени Франциска Скорины }
\centerline { факультет математики и технологий программирования }
\vfill \vfill
\centerline { Лабораторная работа №2 }
\centerline { «Группы, кольца, поля» }
\vfill \vfill
{\leftskip 0.55\hsize \noindent
  Выполнил:\\Хамков Владислав (ПИ-11)\\
  Проверил:\\Васильев Александр Федорович}
\vfill
\centerline { Ноябрь 2021 }
\page

\setuppagenumbering[state=start]
\setuppagenumber[number=1]
\centerline {Задание №1}
\startformula \startalign[align=left]
\NC A = \mathbb{Z},\quad a*b = a + b^2,\quad \forall a,b \in \mathbb{Z} \NR
\NC \forall a,b \in \mathbb{Z} \quad a + b^2 \in \mathbb{Z} \NR
\stopalign \stopformula
Операция $*$ определена на $\mathbb{Z}$, т.е. на $A$.
\startformula \startalign[align=left]
\NC \forall a,b,c \in \mathbb{Z} \quad
    (a*b)*c = (a+b^2)+c^2 = a+b^2+c^2 \ne a+b^2+2bc^2+c^4 = a+(b+c^2)^2 \NR
\stopalign \stopformula
Операция $*$ не ассоциативна,
поэтому $A$ с операцией $*$ полугруппой не является.
Единичного элемента нет.
\vfil


\centerline {Задание №2}
\startformula \startalign[align=left]
\NC M = \{c-d\sqrt{2} \,\, | \,\, c,d \in \mathbb{Z}\}, \quad a*b = a+b, \quad
    \forall a,b \in M \NR
\NC \forall a,b \in M \NR
\NC a*b = a+b = (c_a - d_a\sqrt{2}) + (c_b - d_b\sqrt{2})
    = (c_a + c_b) - (d_a + d_b)\sqrt{2} = c - d\sqrt{2} \in M \NR
\stopalign \stopformula
Операция $*$ определена на $M$. Заметим, что $M \subset \mathbb{R}$. \\
Операция $*$ ассоциативна на $M$ по наследству от $\mathbb{R}$. \\
Операция $*$ коммутативна на $M$ по наследству от $\mathbb{R}$. \\
В $M$ единичным элементом является $0 = 0 - 0\sqrt{2}$ по наследству от $\mathbb{R}$. \\
Для каждого элемента $a$ обратным является $-a = (-c_a) - (-d_a)\sqrt{2}$ по наследству от $\mathbb{R}$. \\
Таким образом, $M$ с операцией $*$ является абелевой группой.
\vfil


\centerline {Задание №3}
\startformula \startalign[align=left]
\NC M = \left{2k+1\ \ |\ \ k\in\mathbb{R}\setminus\left{-\frac12\right}\right}
    = \mathbb{R}\setminus\left{-\frac12\cdot2+1\right}
    = \mathbb{R}\setminus\{0\} = \mathbb{R}^* \NR
\stopalign \stopformula
$M$ не является аддитивный группой, т.к. в ней нет единичного элемента --- $0$.
$M$ является мультипликативный группой, т.к. $M=\mathbb{R}^*$.
\page


\centerline {Задание №4}
\startformula \startalign[align=left]
\NC K = \left{\frac{a}{5^{k-1}}\ \ |\ \ a\in\mathbb{Z},\ k\in\mathbb{N}\right}
    = \left{\frac{5^m\cdot c}{5^{k-1}}\ \ |\ \ c\in\mathbb{Z},\ m,k\in\mathbb{N}\right}
    = \NR\NC = \left{\frac{c}{5^z}\ \ |\ \ c,z\in\mathbb{Z}\right}
    = \left{c\cdot5^z\ \ |\ \ c,z\in\mathbb{Z}\right} \NR
\NC \forall a,b \in K \quad
    a+b = c_a5^z_a + c_b5^z_b
    = 5^{\min{(z_a, z_b)}}\cdot(c_a5^{z_a - \min{(z_a, z_b)}}
        + c_b^{z_b - \min{(z_b, z_b)}})
    = \NR\NC = 5^z\cdot c = c\cdot5^z \in K \NR
\stopalign \stopformula
Операция $+$ определена на $K$. \\
Операция $+$ на $K$ ассоциативна по наследству от $\mathbb{R}$. \\
Операция $+$ на $K$ коммутативна по наследству от $\mathbb{R}$. \\
Элемент $0 = 0\cdot5^0$ является единичным элементом по наследству от $\mathbb{R}$. \\
Для каждого элемента $a$ обратным является $-a = (-c_a)5^z_a$ по наследству от $\mathbb{R}$. \\
Таким образом, $M$ с операцией $+$ является абелевой группой.

\startformula \startalign[align=left]
\NC \forall a,b \in K \quad
    a\cdot b = c_a5^z_a \cdot c_b5^z_b = c_a c_b 5^{z_a + z_b} = c5^z \in K \NR
\stopalign \stopformula
Операция $\cdot$ определена на $K$. \\
Операция $\cdot$ на $K$ ассоциативна по наследству от $\mathbb{R}$. \\
Операция $\cdot$ на $K$ дистрибутивна относительно $+$ по наследству от $\mathbb{R}$. \\
Таким образом, $K$ с операциями $+$ и $\cdot$ является кольцом.
\vfil


\centerline {Задание №5}
\startformula \startalign[align=left]
\NC P = \left{(a,b)\ \ |\ \ a,b\in\mathbb{R} \right}, \quad
    (a,b) + (c,d) = (a,d), \quad (a,b) \cdot (c,d) = (ac, bd) \NR
\NC (1,0) + (0,1) = (1,1) \ne (0,0) = (0,1) + (1,0) \NR
\stopalign \stopformula
Операция $+$ на $P$ не является коммутативной, поэтому $P$ не может быть полем.
\stoptext

\setuppapersize[A4]
\setuplayout[backspace=2cm, topspace=1cm, header=1cm, bottomspace=1cm,
    footer=1cm, width=middle, height=middle]
\setupbodyfont[libertinus, 14pt]
\mainlanguage[ru]
\setuppagenumbering[location={footer,center}]
\definemathmatrix[lmatrix][simplecommand=lmat, left={\left\{\,}, right={\,\right. }]
\definemathmatrix[umatrix][simplecommand=umat, left={\left[\,}, right={\,\right. }]
\definemathmatrix[pmatrix][simplecommand=pmat, left={\left(\,}, right={\,\right) }]
\definemathmatrix[dmatrix][simplecommand=dmat, left={\left|\,}, right={\,\right| }]

%\setupformulas[align=right]
\starttext
\setuppagenumbering[state=stop]
\centerline { Гомельский государственный университет имени Франциска Скорины }
\centerline { факультет математики и технологий программирования }
\vfill \vfill
\centerline { Лабораторная работа №7 }
\centerline { «Определители» }
\vfill \vfill
{\leftskip 0.55\hsize \noindent
  Выполнил:\\Хамков Владислав (ПИ-11)\\
  Проверил:\\Васильев Александр Федорович}
\vfill
\centerline { Декабрь 2021 }
\page

\setuppagenumbering[state=start]
\setuppagenumber[number=1]
\centerline {Задание №1}
\startformula
m = a_{23}a_{4k}a_{j6}a_{5i}a_{11}a_{34}
\stopformula
Для того, чтобы произведение $m$ входило в формулу определителя со знаком минус необходимо и достаточно,
чтобы перестановка $M$ была нечётной.
\startformula
M = \pmat{1,2,3,4,5,j;1,3,4,k,i,6} = \pmat{1,2,3,4,5,6;1,3,4,k,i,6}
\stopformula
\startformula
M = \pmat{1,2,3,4,5,6;1,3,4,k,i,6} \in \left\{ \pmat{1,2,3,4,5,6;1,3,4,2,5,6}, \pmat{1,2,3,4,5,6;1,3,4,5,2,6} \right\}
\stopformula
$(1)(234)(5)(6)$ --- чётная.\\
$(1)(2345)(6)$ --- нечётная.\\
Ответ: $ i=2,\ j=6,\ k=5 $
\vfil

\centerline {Задание №2}
\startformula
|A| = \dmat{3i-1, 2; i, 2i+3} = (3i-1)(2i+3) - 2i = -9 + 5i
\stopformula
\startformula
|B| = \dmat{84434, -84534; 12796, -12896} = \dmat{84434, -100; 12796, -100} = -100(84434 - 12796) = -7164800
\stopformula
\startformula
|C| = \dmat{2, 2, 5; 3, 3, 6; 4, 3, 4} = \dmat{2, 2, 5; 1, 1, 1; 4, 3, 4} = \dmat{0, 0, 3; 1, 1, 1; 4, 3, 4}
= 3\dmat{1,1;4,3} = 3(3-4) = -3
\stopformula
\vfil

\centerline {Задание №3}
\startformula
|F| = \dmat{-2, 1, 3, 4; 4, -1, -1, -9; 2, -1, -2, 0; -2, 2, 9, 6}
= \dmat{-2, 1, 3, 4; 0, 1, 5, -1; 0, 0, 1, 4; 0, 1, 6, 7}
= \dmat{-2, 1, 3, 4; 0, 1, 5, -1; 0, 0, 1, 4; 0, 0, 1, 3}
= \dmat{-2, 1, 3, 4; 0, 1, 5, -1; 0, 0, 1, 4; 0, 0, 0, -1}
= 2
\stopformula
\startformula
|H| = \dmat{-3, 1, 4, 5, -1; 2, 1, 5, -3, 2; 0, -1, 1, 5, -3; 3, -2, 1, 5, 1; 1, 1, -2, 1, 3}
= \dmat{-3, 1, 4, 5, -1; 0, \frac53, \frac{23}3, \frac13, \frac43;
	0, -1, 1, 5, -3; 0, -1, 5, 10, 0; 0, \frac43, -\frac23, \frac83, \frac83}
= \dmat{-3, 1, 4, 5, -1; 0, \frac53, \frac{23}3, \frac13, \frac43;
	0, 0, \frac{28}5, \frac{26}5, -\frac{11}5; 0,0, \frac{48}5, \frac{51}5, \frac45;
	0, 0, -\frac{34}5, \frac{12}5, \frac85}
=
\stopformula
\startformula
|H| = \dots
= \dmat{-3, 1, 4, 5, -1; 0, \frac53, \frac{23}3, \frac13, \frac43;
	0, 0, \frac{28}5, \frac{26}5, -\frac{11}5;
	0, 0, 0, \frac97, \frac{32}7;
	0, 0, 0, \frac{61}7, -\frac{15}{14}}
= \dmat{-3, 1, 4, 5, -1; 0, \frac53, \frac{23}3, \frac13, \frac43;
	0, 0, \frac{28}5, \frac{26}5, -\frac{11}5;
	0, 0, 0, \frac97, \frac{32}7;
	0, 0, 0, 0, -\frac{577}{18}}
= 1154
\stopformula
\vfil

\centerline {Задание №4}
\startformula
|A| = \dmat{1,1,1,1,1; -7,2,1,-4,5; 49,4,1,16,25; -343,8,1,-64,125; 2401,16,1,256,625}
= \dots
\stopformula
Гугл сказал, что $|A|$ --- это определитель Вондермода и считается он вот так:
\startformula
|A| = \dots = (2+7)(1+7)(-4+7)(5+7)(1-2)(-4-2)(5-2)(-4-1)(5-1)(5+4) = -5\cdot 6^8
\stopformula
\vfil

\centerline {Задание №5}
\startformula
|C| = \dmat{2, 2, 5; 3, 3, 6; 4, 3, 4}
= -2\dmat{3,6;4,4} + 3\dmat{2,5;4,4} - 3\dmat{2,5;3,6}
%= -2(12-24) + 3(8-20) -3(12-15)
= 24 - 36 + 9 = -3
\stopformula
\startformula
\startalign \NC
|F| = \dmat{-2, 1, 3, 4; 4, -1, -1, -9; 2, -1, -2, 0; -2, 2, 9, 6}
= 2\dmat{1,3,4;-1,-1,-9;2,9,6} + \dmat{-2,3,4;4,-1,-9;-2,9,6} -2\dmat{-2,1,4;4,-1,-9;-2,2,6} - 0 = \NR\NC
= 2\cdot11 + (-32) -2(-6) = 22 - 32 + 12 = 2
\stopalign
\stopformula
\vfil

\centerline {Задание №6}
\startformula
\startalign \NC
|A| = \dmat{-3, 1, 4, x; 2, x, -1, 0; x, 3, 2, 1; 3, 0, x, 1}
= -\dmat{3, 0, x, 1;-3, 1, 4, x; 2, x, -1, 0; x, 3, 2, 1}
= -\dmat{3, 0, x, 1;-3, 1, 4, x; 2, x, -1, 0; x, 3, 2, 1} = \NR\NC
= -\dmat{3, 0, x, 1; 0, 1, x+4, x+1; 0, x, -\frac23x-1, -23; 0, 3, -\frac13x^2+2, -\frac13x+1}
= -\dmat{3, 0, x, 1; 0, 1, x+4, x+1; 0, 0, -x^2 -4x -\frac23x-1, -x^2 -x -\frac23;
	0, 0, -\frac13x^2 -3x-10, -\frac{10}3x-2} = \NR\NC
= x^4 - 13x^2 - 2x + 14
\stopalign
\stopformula
\vfil

\centerline {Задание №7}
\startformula
|A(x)| = -\dmat{3, 0, x, 1; 0, 1, x+4, x+1;
	0, 0, -x^2 -4x -\frac23x-1, -x^2 -x -\frac23;
	0, 0, -\frac13x^2 -3x-10, -\frac{10}3x-2} =
= x^4 - 13x^2 - 2x + 14
\stopformula
\startformula
|A(0)| 
= 0^4 - 13\cdot0^2 - 2\cdot0 + 14 = 14 \ne 0
\stopformula
\startformula
|A(1)| 
= 1^4 - 13\cdot1^2 - 2\cdot1 + 14 = 0
= -\dmat{3, 0, 1, 1; 0, 1, 5, 2; 0, 0, -\frac{20}3, -\frac83; 0, 0, -\frac{40}3, -\frac{16}3}
\stopformula
\startformula
\dmat{3, 0, 1; 0, 1, 5; 0, 0, -\frac{20}3} = -20 \ne 0
\stopformula
Итого: ранг матрицы --- 4 при $x=0$ и 3 при $x=1$ \\
\vfil

\centerline {Задание №8}
\startformula
A = \pmat{3i-1, 2; i, 2i+3}
\quad \quad
C = \pmat{2, 2, 5; 3, 3, 6; 4, 3, 4}
\stopformula
\startformula
A^* = \pmat{2i+3, -2; -i, 3i-1} \quad |A| = -9 + 5i
\stopformula
\startformula
A^{-1} = \frac{A^*}{|A|} = \frac1{106}\pmat{-33i - 17, 10i + 18; 9i-5, -22i+24}
\stopformula
\startformula
C^* = \pmat{
	\dmat{3,6;3,4}, \dmat{2,5;3,4}, \dmat{2,5;3,6};
	\dmat{3,6;4,4}, \dmat{2,5;4,4}, \dmat{2,3;5,6};
	\dmat{3,3;4,3}, \dmat{2,2;4,3}, \dmat{2,2;3,3}}
= \pmat{-6,7,-3;12,-12,3;-3,2,0}
\quad |C| = -3
\stopformula
\startformula
C^{-1} = \frac{C^*}{|C|} = \pmat{2, -\frac73, 1; -4, 4, -1; 1, -\frac23, 0}
\stopformula
\vfil

\stoptext
